\documentclass{paper}
\usepackage{times}
\usepackage{geometry}
\geometry{letterpaper, portrait, margin=1in}
\usepackage[utf8]{inputenc}
\usepackage{enumitem,amssymb}
\usepackage{ragged2e}
\usepackage{physics}

\usepackage{caption}
\usepackage[hidelinks]{hyperref}
\usepackage{url}

\usepackage{graphicx}
\usepackage{epstopdf}
\graphicspath{{data/cos-w4}}

\usepackage{biblatex}
\bibliography{refs}

\title{cos-w4 questions}
\author{ariahayd}
\date{22 February 2022}

\begin{document} 

\maketitle

\begin{enumerate}
    \item % 1. 
      For the Friedmann equation as a solution Einstein field equations for
      a homogeneous and isotropic universe,

      \begin{equation}
        \dot{R}^2 = \frac{8 \pi G}{3}\rho R^2 - k c^2 + \frac{\Lambda}{3} R^2
        \label{eq:friedmann}
      \end{equation}

      Solve for R for each case:

      \begin{enumerate}
        \item
          The Milne model, \(k < 0, \rho = 0, \Lambda = 0\):
          \[ \dot{R}^2 = k c^2 \implies R = t \sqrt{kc^2} \]

        \item
          The Einstein-de-Sitter model, \(k = 0, \Lambda =0\), where 
          \(\rho \propto R^{-3}\) for matter-dominated density:

          \[ 
          \dot{R}^2 = \frac{8 \pi G}{3}\rho R^2 \implies
          \dot{R} = 
          \left[\frac{8 \pi G}{3}(\propto R^{-3})\right]^{\frac{1}{2}} R
          \] 

          \[
          \int R^{\frac{1}{2}}\dd{r} = \int \sqrt{\frac{8 \pi G}{3}}\dd{t}
          \implies \frac{2}{3}R^{\frac{3}{2}} 
          = \left(\frac{8 \pi G}{3}\right)^\frac{1}{2} t + \mathcal{C} 
          \]

        \item
          A flat geometry in the radiation-dominated era, 
          \(k = 0, \Lambda = 0 \) where \(\rho \propto R^{-4}\):

          \[ 
          \dot{R}^2 = \frac{8 \pi G}{3}\rho R^2 \implies
          \dot{R} = 
          \left[\frac{8 \pi G}{3}(\propto R^{-4})\right]^{\frac{1}{2}} R
          \] 

          \[
          \int R\dd{r} = \int \sqrt{\frac{8 \pi G}{3}}\dd{t}
          \implies \frac{1}{2}R^{2} 
          = \left(\frac{8 \pi G}{3}\right)^\frac{1}{2} t + \mathcal{C} 
          \]

        \item
          Parameters dominated by \(\Lambda\), \(\rho 
          \rightarrow 0, k=\mathcal{C}\):

          \[
            \int R^{-2}\dd{r} = \int \sqrt{\frac{\Lambda}{3}}\dd{t}
          \implies -\frac{1}{R} = \sqrt{\frac{\Lambda}{3}} t + \mathcal{C}
          \]

      \end{enumerate}


    \item % 2.
      \begin{enumerate}
        \item
          For the case of a matter-dominated Einstein-de-Sitter cosmology,
          show expansion follows time as:
          \[ t = \left(\frac{1}{6 \pi G \rho}\right)^{0.5} \]

          The cosmology is derived from (b) above:
          \[ 
          \dot{R}^2 = \frac{8 \pi G}{3}\rho R^2 \implies
          \dot{R} = 
          \left[\frac{8 \pi G}{3}(\rho \propto R^{-3})\right]^{\frac{1}{2}} R
          \] 

          \[
          \int R^{\frac{1}{2}}\dd{r} 
          = \int \sqrt{\frac{8 \pi G \rho}{3}}\dd{t}
          \implies 
          \frac{2}{3}R^{\frac{3}{2}} \propto
          \left(\frac{8 \pi G \rho }{3}\right)^\frac{1}{2} t 
          \]

          \[
            R^3=\left(\frac{9}{4}\times
            \frac{8 \pi G \rho}{3}\right)^\frac{1}{2} t 
          \implies
          t = \left(\frac{1}{6 \pi G \rho}\right)^\frac{1}{2} R^3
          \]

        \item
          For the case of a radiation-dominated energy density \(\rho\),
          show expansion follows time as:
          \[ t = \left(\frac{3}{32 \pi G \rho}\right)^{0.5} \]

          The cosmology is derived from (c) above:
          \[ 
          \dot{R}^2 = \frac{8 \pi G}{3}\rho R^2 \implies
          \dot{R} = 
          \left[\frac{8 \pi G}{3}(\rho \propto R^{-4})\right]^{\frac{1}{2}} R
          \] 

          \[
          \int R\dd{r} 
          = \int \sqrt{\frac{8 \pi G \rho}{3}}\dd{t}
          \implies 
          \frac{1}{2}R^2 \propto
          \left(\frac{8 \pi G \rho }{3}\right)^\frac{1}{2} t 
          \]

          \[
          R^2 =\left(4\times\frac{8 \pi G \rho}{3}\right)^\frac{1}{2} t 
          \implies
          t = \left(\frac{3}{32 \pi G \rho}\right)^\frac{1}{2} R^2
          \]

      \end{enumerate}

    \item % 3.
      The complete Friedmann equation is expressed in Equation 
      \ref{eq:friedmann}.  Disregarding expansion due to \(\Lambda\) and 
      considering only the flat case where \(k=0\) as the equilibrium point 
      for critical density \(\rho_c\), the expression simplifies to:

      \[ \dot{R}^2 = \frac{8 \pi G}{3}\rho_c R^2 \]

      Where \( H:= \frac{\dot{R}}{R} \), the present day scale length of
      expansion can be written as:

      \[ 
      \frac{\dot{R}^2}{R_0^2} = \frac{8 \pi G}{3}\rho_c 
      \implies
      \rho_c = \frac{3 H_0^2}{8 \pi G}
      \]

      \(\rho_c\) evalutes to \(8.68\times10^{-27} kg/m^{3}\) 
      for \(H_0 = 68 km/s/Mpc\).

    \item % 4.
      The ideal gas law relates pressure to temperature.
      \[ PV = n\mathcal{R}T \implies P = \rho\mathcal{R}T\] 

      Equate pressures between an ideal hydrogen gas and the cosmological
      equation of state \(P = \omega \rho c^2\):
      \[ 
      \omega \rho c^2 = \rho \mathcal{R} T 
      \implies 
      \omega = \frac{\mathcal{R}}{c^2}T = \mathcal{C}T
      \]

      So, \(\mathcal{C} = \frac{\mathcal{R}}{c^2} \). This relationship is 
      valid for the temperature domain of atomic hydrogen as an ideal gas: 
      \[ 20 K < T < 5000 K \]

    \item % 5.
      The energy density of the cosmic microwave background radiation follows
      a black body spectrum which can be scaled to different peak wavelengths 
      and temperatures through Wien's displacement law, which maintains the 
      profile of the spectrum across peak radiation intensity as 
      \(\lambda \propto T^{-1}\) for a spectrum given by:
      \begin{equation}
        \dd{u} = \frac{8 \pi h \nu^3}{c^3} \frac{1}{
          \exp(\frac{h \nu}{kT}) - 1} \dd{\nu}
        \label{eq:spectrum}
      \end{equation}

      The peak of the black body spectrum rises increases with \(\nu\) and 
      declines again with the inflection of the exponential term:
      \[ \exp(\frac{h \nu}{k T}) > 1 \]

      Substituting \(\nu = \frac{c}{\lambda}\) shows the decline of the black 
      body spectrum in wavelengths greater than the thermal wavelength:
      \begin{equation}
        \lambda = \frac{h c}{k T}
        \label{eq:lambda}
      \end{equation}

      The observed thermal wavelength, \(\lambda^\prime\) can be 
      dependendent on \(z\) for significant redshifts:
      \[ \frac{\lambda^\prime(z)}{\lambda_{emitted}} = (1 +z)
      \implies \lambda^\prime = (1 + z)\lambda_{emitted} \]

      The characteristic temperature of a black body spectrum shifted to
      significant \(z\) can then be substituted into Equation \ref{eq:lambda}:
      \[ T^\prime = \frac{h c}{k \lambda^\prime} 
      = \frac{h c (1 + z)}{k \lambda_{emitted}}= T (1 + z)\]

\end{enumerate}

This paper is available publicly.\cite{Hayden_Cosmology_Source_Repo}

\pagebreak
\printbibliography

\end{document}

