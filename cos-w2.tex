\documentclass{paper}
\usepackage{times}
\usepackage{geometry}
\geometry{letterpaper, portrait, margin=1in}
\usepackage[utf8]{inputenc}
\usepackage{enumitem,amssymb}
\usepackage{ragged2e}
\usepackage{physics}
\usepackage{caption}
\usepackage{tikz}
\usepackage{graphicx}
\usepackage{epstopdf}

\graphicspath{{data/cos-w2/}}

\title{cos-w2 questions}
\author{ariahayd}
\date{8 February 2022}

\begin{document} 

\maketitle

\begin{enumerate}
    \item % 1. 
      \begin{enumerate}
        \item
          Gavitational mass is the property of an object that affects the
          local gravitional field. This mass is the active mass in a mass-mass
          interaction in the Newtonian model.
        \item
          Inertial mass is separate property of an object the responds to
          the local gravitational field. The inertial mass is passive, or 
          responsive, in the Newtonian forces model between two mass in a 
          gravitational interaction. The inertial mass acts to dampen the
          acceleration of an object in a gravitional field.
      \end{enumerate}

      There are objections to the simple model for two-body interactions
      under Newtonian gravity:
      \begin{enumerate}
        \item
          Observationally, the elliptical orbit of Mercury was seen to
          precess: the semi-major axis rotated over many orbits. This meant
          that some component of the inertial frame containing the 
          Mercury-Sun system was rotating and that the Mercury is not a simple 
          two-body interaction about the center of mass of the Sun and planet.
        \item
          Candidate absolute inertial frames (a frame on the surface of the
          Earth, a between bodies in the Solar System, etc.) are all rotating
          in orbits among larger systems, which average to the Newtonian 
          inertial frame, but contribute accelerations to the systems. 
      \end{enumerate}
      
      The Newtonian model for gravity is consistent for inertial frames, local
      areas for interactions without any component of acceleration from 
      outside of the frame. Frames must be non-accelerating relative to 
      each other for Newtonian interactions. Mach's Principle is
      the generalization that Newton's Laws are valid for inertial frames 
      relative to each other rather than to a static background of no 
      acceleration in empty space.


    \item % 2. 
      lore

    \item % 3.
      lore

   \item % 4.
     For a measured radius $a$ across a spherical surface of radius $R$,
     the geometry is drawn in Figure \ref{fig:sphere}. The flattened radius
     is periodic with $sin \left(\frac{R}{a}\right)$. Demonstrate 
     (\ref{eqn:curvature}) simplifies to $k = 1/R^2$ where 
     $C=2 \pi R sin \left(\frac{a}{R}\right)$.

     \begin{equation}
       k = \frac{3}{\pi} \lim_{a \to 0} \left( \frac{2 \pi a - C}{a^3} \right)
       \label{eqn:curvature}
     \end{equation}

     \begin{figure}[!htb]

       \begin{tikzpicture}
         % 2-sphere surface
         \draw (0,0) circle (2cm);
         % perspective-facing circumference
         \draw (-1.404,1.404) arc (180:360:1.404 and 0.2);
         % perspective-hidden circumference
         \draw[dashed] (-1.404,1.404) arc (0:180:-1.404 and 0.2);
         % measured radius
         \draw (0,2) arc (300:360:-.9) node[midway]{$a$};
         % flattened radius
         \draw[dashed] (0,1.404) -- node{\tiny $R sin 
           \left(\frac{a}{R}\right)$}(1.404,1.404);
        % sphere radius points
         \fill[fill=black] (0,0) circle (1pt);
         \fill[fill=black] (0,2) circle (1pt);
         \fill[fill=black] (1.414,1.414) circle (1pt);
         % projection of triangle
         \draw[dashed] (0,0) -- (0,2);
         \draw[dashed] (.702,0) (0:60);
         \draw[dashed] (0,0) -- node[below]{$R$} (1.414,1.414);
       \end{tikzpicture}

       \caption{2-sphere of radius $R$ on which a circle of radius $a$ will be
         measured to have radius $a$, which has a flattened radius of
         $R sin \left( \frac{a}{R} \right) \neq a$}
       \label{fig:sphere}
     \end{figure}

     Substitute $C$:
       \[ k = \frac{3}{\pi} \lim_{a \to 0} \left( \frac{2 \pi a - 
         2 \pi R sin \left(\frac{a}{R}\right)}{a^3} \right) \]

     Expand $sin \left(\frac{a}{R}\right)$ with the Maclaurin series:
       \[ sin \left(\frac{a}{R}\right) = \left( \frac{a}{R} 
         - \frac{a^3}{R^3 3!} + \frac{a^5}{R^5 5!} \right) \]

     Gives:
      \[ k = \frac{3}{\pi} \lim_{a \to 0} \left( \frac{2 \pi a}{a^3} - 
        \frac{2 \pi R}{a^3}\ \left( \frac{a}{R} - \frac{a^3}{R^3 3!} +
        \frac{a^5}{R^5 5!} \ldots \right) \right)\]

     Which simplifies under the limit $a \to 0$:
      \[ k = \frac{3}{\pi} \lim_{a \to 0} \left( \frac{2 \pi a}{a^3} - 
        \frac{2 \pi a}{a^3} + \frac{2 \pi }{R^2 3!} - 
        \frac{2 \pi R a^5}{a^3 R^5 5!} \ldots \right) 
        = \frac{3}{\pi} \frac{2 \pi}{R^2 3!} \lim_{a \to 0}
        \left( > |a^2| \right) = \frac{1}{R^2} \]


   \item % 5.

\end{enumerate}

\pagebreak
\bibliographystyle{plain}
\bibliography{refs}

\end{document}

