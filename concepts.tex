\documentclass{paper}
\usepackage{times}
\usepackage{geometry}
\geometry{letterpaper, portrait, margin=1in}
\usepackage{setspace} \doublespacing
\usepackage[utf8]{inputenc}
\usepackage{enumitem,amssymb}
\usepackage{ragged2e}
\usepackage{physics}
\usepackage{siunitx}
\usepackage{wasysym}
\sisetup{separate-uncertainty=true}
\usepackage{float}
\usepackage{mathtools}
\usepackage{amsmath}

\usepackage{caption}
\usepackage[hidelinks]{hyperref}
\usepackage{url}

\usepackage{graphicx}
\usepackage{epstopdf}
\usepackage{tikz}
\graphicspath{{data/concepts}}

\usepackage{aas_macros}
\usepackage[style=authoryear]{biblatex}
\bibliography{refs} %refs.bib
\nocite{*}

\immediate\write18{texcount \jobname.tex -out=\jobname.sum}

% set frameboxes to be borderless
\setlength{\fboxsep}{0pt}
\setlength{\fboxrule}{0pt}

\begin{document} 

\title{The constituents of the Universe and the growth of structure}
\author{Aaron Hayden}
\date{April 24, 2022}
\maketitle

\section*{Overview}
  The Universe is shaped by its constituent energies, and its constituents are
  shaped by the size of space. The evolution of the Universe is understood 
  through the standard model of Cosmology, developed over the last century of 
  observations of matter and energy on large scales, of structures in cubic 
  gigaparsec volumes space, and on small scales where the quantum mechanical 
  properties of energy fields emerge. 

  The most successful model is $\Lambda$CDM, the concordance model, with 
  dynamics explained through General Relativity and Quantum Field Theory, with 
  contributions from dark energy ($\Lambda$) and cold dark matter (CDM). These 
  last two components are observationally confirmed but not entirely explained 
  by theory. 

  The densities of energy in the fields within the Universe determine the 
  geometry of space.  Observations indicate that the curvature of the metric 
  of spacetime over large distances is nearly flat and has been throughout 
  most of the history of the Universe.
  
  $\Lambda$CDM is based on a Hot Big Bang (HBB) origin of high energy density 
  that has since evolved to today's Universe of lower density.  Principal 
  evidence of a HBB is from a homogeneous distribution of quasar radio signals 
  observed isotropically throughout space, but all distant in time, indicating 
  that the environments of galaxies were very different in the past 
  (\cite{Secrest_2021}) and that the Universe is not in an eternal steady 
  state. 

  The standard model explains how energy fields latent in space express in 
  different ways across evolutionary epochs, expanding the scale of space in 
  the inflationary and current epochs, and smoothing out the distribution of 
  gravitationally bound particles in the radiation dominated era. The main 
  success of $\Lambda$CDM is the accounting of the energies that come from 
  these fields, as shown in Fig. \ref{fig:Intro-constraints}. 

  \begin{figure}[H]
    \begin{centering}
    \includegraphics[scale=0.7]{Intro-constraints.pdf}
    \caption{The Sloan Digitial Sky Survey updated data release, including 
      measurements of the cosmic microwave background (CMB) from the Planck,
      mission places constraints on the parameters of $\Lambda$CDM, including
      energy from pressureless CDM, photons, baryons, neutrinos, and dark
      energy.
    Credit: \cite{2021PhRvD.103h3533A}}
    \label{fig:Intro-constraints}
    \end{centering}
  \end{figure}

\section*{Dark Matter}
  Large, gravitationally bound systems such as galaxies and galaxy clusters 
  appear embedded in gravitational potentials deeper than expected given the 
  masses observed interacting within the system. The depth of these potentials 
  is inferred due to unseen, \textit{dark} matter.

  Early evidence for dark matter in the halo of a galaxy comes from
  measurements of the rotational velocities of the material in the arms of
  spiral galaxies. A spiral galaxy is a stable, virialized structure bound by
  gravity and is expected to follow the Newtonian gravitation model for a
  fluid rotating around a center mass. The spiral arms are not coherent
  structures in themselves, but are the result of a density way propagating 
  through the fluid of the galactic disk. 

  The rotational velocity of the disk can be estimated via the width of the 
  HII emission lines across the disk of the galaxy, as was done for M31 in 
  Fig. \ref{fig:DM-masscurve}, showing the velocity field follows a profile more 
  like a rotating solid body a fluid swirling around a core. Because galaxies 
  can indeed be modeled hydrodynamically, the only explanation for a velocity 
  field continuing beyond the visible matter of the galaxy is for an 
  unobservable mass encircling the galaxy in a halo.

  \begin{figure}[H]
    \begin{centering}
    \includegraphics[scale=0.4]{DM-masscurve.pdf}
    \caption{Fourteen different mass models fitting the velocity map data of
      M31 are combined to yield a band of mass estimates, on the left, and 
      surface densities, on the right, from the center to beyond the luminous 
      edge of M31, a galaxy with an has an apparent angular diameter of 
      \SI{178}{arc-minutes}.
    Credit: \cite{1970ApJ...159..379R}}
    \label{fig:DM-masscurve}
    \end{centering}
  \end{figure}

  The projected mass sheets of these halos are responsible for strong and weak 
  gravitational lensing. Since these early observations, dark matter halos 
  have been measured for most bound systems on the scale of galaxies or 
  clusters, though there is also evidence of anomalous galaxies that may lack 
  a significant concentration of dark matter around them 
  (\cite{10.1093/mnras/stab3491}). 

  Halos have been demonstrated to move with core material of galaxies and not 
  with energetic plasma that makes up majority of baryonic mass in a galaxy 
  when the plasma is disrupted by ram-pressure stripping.  Dark matter is 
  bound to the gravitational potential of the galaxy rather than to the 
  baryons in the potential (\cite{Clowe_2006}). This observation reinforces 
  the \textit{cold} nature of dark matter; it cannot kinetically escape the 
  potential that binds baryonic masses, indicating that it is decoupled from 
  interactions with baryons. 

  That dark matter could not be constituted entirely from baryonic particles
  follows from solutions to the HBB model of the radiation era that show 
  pressure waves traveling through the irradiated plasma damping the accretion 
  of the baryonic matter in any potential wells evident in the anisotropy of 
  spherical harmonic modes in the CMB.

  The baryon acoustic oscillation (BAO) signal comes from modeling the 
  destructive interference pattern of density waves in the coupled 
  photon-baryon fluid that permeated the radiation-dominated era of the 
  Universe. The restoring force of a pressure variance in this fluid comes 
  from the ionization and scattering between primordial atoms and photons 
  interacting frequently due to the overwhelming prevalence of high energy 
  photons resulting from baryogenesis, when most baryons annihilated with 
  their anti-particle pair, converting to photons.

  Gravitational fluctuations must have been built prior to the decoupling and 
  could not have been solely due to density variations in the photon-baryon 
  fluid.  There must have been a dark matter component ``frozen'' into the 
  environment of the plasma, seeding the gravitational fluctuations that would 
  quickly accrete material after decoupling, forming galaxies.  As shown in 
  Fig. \ref{fig:DM-BAO}, the number density of galaxies in volumes of space, 
  measured through redshift surveys, correlates to the BAO signal at the scale 
  of an earlier period in the Universe.

  Dark matter can be characterized as two types, either due to massive, 
  compact halo objects (MACHOs) or by weakly interacting massive particles 
  (WIMPs). Searches for the lensing events from MACHOs in a galactic halo have 
  returned no significant findings. Though dark matter can be observed through 
  phenomena understood in the standard model, there is no identified candidate 
  WIMP particle that could constitute dark matter. An explanation of the WIMP 
  contribution to dark matter from something like a primordial, massive 
  graviton, requires new physics (\cite{PhysRevLett.128.081806}).

  \begin{figure}[H]
    \begin{centering}
    \includegraphics[scale=0.5]{DM-BAO.pdf}
    \caption{A combination of two galaxy surveys shows the correlation of
      the density of galaxies with modeled density distribution from the
      BAO wave modes in the coupled photon-baryon
      fluid of the radiation dominated era. $k$ is the wave number of 
      an oscillating pressure wave. \(h = 0.7\) is the dimensionless Hubble
      constant. These observations match a model with a total energy density
      due to matter, \(\Omega_m = 0.25\), while the baryonic contribution is
      observed to be \(\Omega_b = 0.05\), indicating that the 
      difference is dark matter (\cite{Eisenstein_2005}).
    Credit: \cite{2007MNRAS.381.1053P}}
    \label{fig:DM-BAO}
    \end{centering}
  \end{figure}

\section*{Baryons}
  Constraints on the density of baryons among the energies of the Universe are
  based on modeling the prevalence of atoms generated during the big bang 
  nucleosynthesis (BBN) epoch or by modeling angular size of spherical 
  harmonic patterns in the surface of last scattering of decoupling. Both 
  approaches for constraining the baryonic contribution to the matter density 
  are indirect because the methodologies infer the baryonic prevalence from 
  understood physics at early times rather than from direct measurements.

  Particles coalesced in at least four distinct epochs. The end of the first 
  era, the X-Boson Era, at \SI{E-38}{s} from HBB, fixed the baryon and lepton 
  number.  In the second, the Quark-Hadron Era, quarks combined to form 
  baryonic nucleons. This era ends at \SI{E-6}{s} with an asymmetry between 
  baryonic particle and antiparticle pairs, in which normal baryons are 
  estimated with an excess of \SI{E-9} per pair (\cite{1993PhRvL..70.2833F}). 
  The third, the Lepton Era, ends at \SI{1}{s} as the temperature decreases to 
  where neutrinos are no longer coupled to electron-positron pairs, and can 
  escape reactions. Throughout these eras, WIMPs are expected to have been 
  present in the background contributing to local density fluctuations 
  through gravitation, but not significantly through coupled interactions.

  The fourth epoch, the Nucleosynthesis Era, between \SI{1}{s} and 
  \SI{300}{s}, was the duration over which weak force interactions changed
  the frequency distribution of protons and neutrons and primordial atomic 
  nuclides formed. The prevalence of free neutrons was constrained by both
  processes, and the resulting abundances of light elements throughout the 
  Universe generally matches the Standard BBN (SBBN) prediction. SBBN is based 
  on the foundational physics of $\Lambda$CDM during the radiation dominated 
  era and the inclusion of three neutrino species (\cite{Cyburt_2016}). 

  Weak interactions maintained the balance of protons and neutrons early in
  the Nucleosynthesis Era until the declining temperature of the fluid reduced 
  the likelihood of producing the neutron, slightly heavier than the proton, 
  within the reaction balance. At \(T = \SI{E10}{K}\), the fraction of 
  neutrons to baryons froze at \(18\%\). This represents the neutron budget 
  for further nuclear reactions that produce primordial atomic nuclei, a 
  budget declining exponentially over the next \SI{886}{s} following the 
  characteristic timescale for the decay of free neutrons through \(\beta\)
  emission.

  Atomic nuclei do not grow through many-body reactions because the cross 
  section of interaction for several specific particle ingredients needed to 
  form a nucleus is rare. Instead, nuclei are built through a chain of 
  two-body reactions. The first reaction in the chain combines a proton with a 
  neutron to form deuterium. However, a deuterium nucleus is not tightly bound 
  and can easily be broken back into constituent particles by interactions 
  with photons until temperature of \(T = \SI{E9}{K}\), which happened at 
  \(t = \SI{300}{s}\). After this point in cooling, deuterium nuclei were 
  available to form \(\prescript{3}{}{He}\), \(\prescript{4}{}{He}\), 
  \(\prescript{6}{}{Li}\), \(\prescript{7}{}{Li}\), and 
  \(\prescript{7}{}{Be}\) through chains that reach equilibrium on timescales 
  shown in Fig. \ref{fig:BBN-frac}.

  \begin{figure}[H]
    \begin{centering}
    \includegraphics[scale=0.3]{BBN-frac.pdf}
    \caption{A chart of the mass fraction of baryons and nuclides over the
      SBBN era. Note that the fraction of protons remains essentially static
      while neutrons are consumed to make nuclides. The mass fraction of 
      neutrons declines overall, but drops precipitously after 
      \(t = \SI{200}{s}\) as the neutron number fraction among baryons froze 
      and the characteristic lifetime of unbound neutrons allowed rapid decay.
      The mass fractions of primordial nuclides heavier than 
      \(\prescript{4}{}{He}\) are vanishingly small.
    Credit: \cite{ryden2003introduction}}
    \label{fig:BBN-frac}
    \end{centering}
  \end{figure}

  Measuring the metal content of stars in the least active globular clusters, 
  which are presumed to be the best preserved early environments that can be 
  observed nearby, approximates but does not exactly match the predicted 
  abundances of SBBN atoms. This indicates some stellar processing prior to 
  the oldest clusters \cite{Kalirai_2010}. Using clusters as ``chronometers'' 
  suggests a time since the HBB of 
  \SI{15.6(46)E9}{years} (\cite{1999ApJ...521..194C}).
  Fig. \ref{fig:BBN-ratios} shows the observed abundances within predicted
  ranges.

  Most of the baryons in the Universe, 75\%, are not gravitationally bound in 
  structures like galaxies, but are diffuse in the intergalactic medium. By 
  measuring the dispersion, the delay in repeated signaling of fast radio 
  bursts through a modeled column of electrons relative to a true vacuum with 
  no particles to scatter the photons, the unbound baryonic matter has been 
  measured at \(\Omega_b = 0.051\) (\cite{2020Natur.581..391M}).

  \begin{figure}[H]
    \begin{centering}
    \includegraphics[scale=0.4]{BBN-ratios.pdf}
    \caption{A modeled prediction of the abundances of nuclides through big
      bang nucleosynthesis assuming a neutron characteristic lifetime of
      \(\tau_n = \SI{880.3(11)}{s}\) and a neutrino species number of 3. 
      \(Y\) is the fraction of primordial \(\prescript{4}{}{He}\). The width 
      of each band is related to the uncertainty in the cross-section of 
      interaction for each nuclear species. Boxes in the figure are values 
      allowed by observations. The vertical striped band is the constraint 
      based on observations of deuterium, limiting 
      \(0.043 \leq \Omega_b \leq 0.051\), while the crosshatched band 
      is the inference of \(\Omega_b\) from the CMB. Note that values for 
      \(\prescript{7}{}{Li}\) do not match observation. The baryon to photon 
      ratio is a range of excess photons resulting from baryon pair 
      production.
      Credit: \cite{liddle2015introduction}}
    \label{fig:BBN-ratios}
    \end{centering}
  \end{figure}

\section*{Dark Energy}
  The energy density of the Universe seems to drive towards a flat spatial 
  metric even at points in the the standard model evolution of the Universe 
  which could be expected to deviate from the critical density due to the 
  shaping effects of energy densities.

  The pressure nudging spacetime to a flat curvature in the current epoch is 
  known as \textit{dark} energy, the $\Lambda$ component of the $\Lambda$CDM
  model, because the source of energy has not yet been explained.  It is 
  thought to be a constant, latent energy density of the vacuum of spacetime 
  in a phase above a potential ground state.

  Evidence for dark energy in the $\Lambda$CDM model comes from two 
  significant observations that independently constrain the contribution of 
  $\Lambda$ to the energy budget, namely measurements of the recession of 
  supernovae as standard candles, shown in Fig. \ref{fig:DE-sn_lightcurve}, 
  and measurements of the angular size of standard rulers in the CMB 
  anisotropy, in Fig. \ref{fig:DE-MCMC}.

  Assuming redshift to be a proxy for distance, the Hubble diagram in Fig. 
  \ref{fig:DE-sn_lightcurve} shows an increase in recession velocity with 
  distance, implying an acceleration between the positions of observation and 
  event against the rest frame of a fundamental observer. The Supernova 
  Cosmology Project later added ten more supernova events to this diagram, 
  maintaining the trend (\cite{2012ApJ...746...85S}).

  \begin{figure}[H]
    \begin{centering}
    \includegraphics[scale=0.4]{DE-sn_lightcurve.pdf}
    \caption{The figure on the top is a fit of measurements of the multicolor 
      light curve shape (MLCS) characteristic of SNe Ia as a function of 
      measured redshift of the event spectrum. The figure below is the
      difference between measured magnitude and modeled magnitude for an 
      environment with no dark energy to show that observations convincingly 
      deviate from an \(\Omega_{\Lambda} = 0\) environment.
      Credit: \cite{Riess_1998}}
    \label{fig:DE-sn_lightcurve}
    \end{centering}
  \end{figure}

  \begin{figure}[H]
    \begin{centering}
    \includegraphics[scale=0.5]{DE-MCMC.pdf}
    \caption{Constraints on the free parameters of the $\Lambda$CDM
      model predicting a range of values for \(\Omega_{\Lambda}\). The contour 
      lines of each parameter represent confidence intervals in the 
      measurement. The values for \(\Omega_{\Lambda}\) are projected from
      Markov chain Monte Carlo (MCMC) simulations of inputs from Wilkinson 
      Microwave Anisotropy Probe (WMAP) data across the parameter space 
      (\cite{Bennett_2011}). For inputs in the domain of likely values for 
      \(\Omega_b\) and \(\Omega_c\) (CDM), the models independently agree at 
      \(\Omega_{\Lambda} \approx 0.73\).
    Credit: \cite{liddle2015introduction}}
    \label{fig:DE-MCMC}
    \end{centering}
  \end{figure}

  Alternatives to the $\Lambda$CDM model include modifications to the equation 
  of state for dark energy, assumed to have a constant value of 
  \(w = \frac{P}{\rho c^2} = -1\) for the modern epoch.  Instead, the energy 
  density is fixed to the grid of spacetime and expressed as a pressure on 
  volumes of vacuum that becomes more significant where the matter density 
  drops, as it has in the current epoch, to \(w = -\frac{1}{3}\). This model 
  is called \textit{quintessence}, is considered to be distinct from a 
  cosmological constant, but with a similar effect on observations
  (\cite{PhysRevLett.80.1582}). 

  In this modification to $\Lambda$CDM the source of pressure from dark energy 
  is a latent scalar field coupled to the geometry of the Universe, resulting 
  in spacetime growth under an \(R_h = ct\) model, and may currently fit 
  observations best when compared across models and input parameters 
  (\cite{10.1093/mnras/sty1962}).

\section*{Growth of Structure}
  Galaxies and galaxy clusters are the largest gravitationally bound 
  structures directly observed. Because a galaxy is a bound object, it may be 
  assumed to have not changed size with cosmological inflation, so would have 
  the same size today as when it stabilized.  The era that separated one 
  galaxy from the constant background density of the Universe occurred at a 
  redshift of 100.  The model for collapse of the background density into a 
  bound, virialized structure is the Jeans criteria and is highly dependent on 
  the thermal velocity and mass density of a fluid (\cite{Jeans1902}).

  \begin{figure}[H]
    \begin{centering}
    \includegraphics[scale=0.4]{Struct-mass.pdf}
    \caption{A graph of the minimum mass needed to collapse into a virialized
      structure over time. Masses in the light, unshaded region can collapse 
      and become luminous.  The dashed lines represent constant 
      post-virialization temperatures of \SI{E4}{K} and \SI{E3}{K}.  The 
      darkest shaded region cannot cool through radiative means because the 
      temperature after virializing would be the same as the cosmic microwave 
      background. The solid line is drawn to show about where 
      \(\Omega_b \times \SI{2E6}{M_{\astrosun}}\) can collapse at \(z=30\).
      Credit: \cite{1997ApJ...474....1T}}
    \label{fig:Struct-mass}
    \end{centering}
  \end{figure}

  The minimum mass for gravitational collapse after the radiation dominated
  era, with an initial temperature of \(T \approx \SI{3000}{K}\), is on the 
  order of \SI{E5}{M_{\astrosun}}, shown in Fig. \ref{fig:Struct-mass}.  This 
  is lower than the masses of galaxies and about similar to the mass of 
  globular clusters.  The Jeans criteria indicates that galactic masses could 
  certainly have coalesced into virialized structures in the time since 
  decoupling.

  The photon-baryon fluid plasma of the early Universe was concurrent with a 
  field of dark matter particles that, through fluctuations in density 
  distribution, created the initial seeds for structural growth. The plasma 
  was not coupled to the dark matter other than through gravitation, causing
  a feedback loop in the evolution of the fields. A smoothness in the density
  of the plasma was enforced through electromagnetic interactions between the 
  ions and photons.  This smoothness prevented the plasma from accreting into 
  the gravitational potentials of the dark matter. In turn, the dark matter 
  could not further collapse into the potential wells due to pressure from the 
  fluid, the Silk damping effect.  

  Two thermodynamic models explain how the plasma would have evolved through 
  the radiation-dominated era. Hierarchical formation assumes an adiabatic 
  fluid that formed large scale structure before splitting into smaller scale 
  structures. Bottom-up formation begins with structures on small scales that 
  grow into larger structures, requiring the photon-baryon fluid accrete into 
  the local gravitational potentials without damping. In both models, 
  accretion requires longer timescales than is possible since decoupling, 
  implying a dark matter precursor to density fluctuations.

  The Millennium Simulation modeled the hierarchical evolution of large clouds 
  of CDM interacting only though gravitation from a ``glasslike'' Gaussian 
  distribution just after decoupling to the current era and found that growth 
  in the CDM filaments results in structures remarkably similar to what is 
  observed in the Universe today (\cite{desjacques2018large}). Fig. 
  \ref{fig:Struct-power_spectrum} shows the power spectrum of features in the 
  density distribution that can be compared to observable redshifts
  (\cite{2005Natur.435..629S}). 

  \begin{figure}[H]
    \begin{centering}
    \includegraphics[scale=0.4]{Struct-power_spectrum.pdf}
    \caption{The dimensionless power spectrum at later redshifts of the
      Millennium Simulation model. Where \(\Delta^2\) approaches unity, 
      nonlinear growth takes over. Solid lines represent linear growth in the 
      Poisson distribution. The dashed line is the shot noise limit of the 
      model due to the number of bodies in the statistics.
      Credit: \cite{2005Natur.435..629S} Supplement}
    \label{fig:Struct-power_spectrum}
    \end{centering}
  \end{figure}

\section*{Conclusions}
  $\Lambda$CDM explains much of the fossilized evidence of the HBB origin, the 
  features embedded in the CMB anisotropy and the abundances of primordial 
  elements in a Universe that is undergoing further nuclear processing. Dark 
  matter and dark energy have an unexplained origin in the current theory.

  Data from the Planck mission may indicate that the Universe has a closed 
  geometry after all, which challenges the inflation solution for the horizon
  and flatness problems (\cite{2020NatAs...4..196D}). Resolving the curvature 
  of spacetime is a fundamental test of the validity of $\Lambda$CDM. 

  Further tests will attempt accounting for the dark sources of energy density 
  in a way that satisfies the model. Physicists are seeking to characterize 
  the WIMP through direct detection (\cite{PETER201445}) and measure 
  variations and sources of dark energy to determine if modifications to 
  $\Lambda$CDM like quinessence are more reflective of the Universe.

%TC:ignore

\pagebreak
\begin{singlespace}
\printbibliography
\end{singlespace}

%TC:endignore

\end{document}

