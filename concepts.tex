\documentclass{paper}
\usepackage{times}
\usepackage{geometry}
\geometry{letterpaper, portrait, margin=1in}
\usepackage[utf8]{inputenc}
\usepackage{enumitem,amssymb}
\usepackage{ragged2e}
\usepackage{physics}

\usepackage{caption}
\usepackage[hidelinks]{hyperref}
\usepackage{url}

\usepackage{graphicx}
\usepackage{epstopdf}
\usepackage{tikz}
\graphicspath{{data/concepts}}

\usepackage{biblatex}
\bibliography{refs} %resfs.bib

\title{cosmology concepts}
\author{ariahayd}
\date{\today}

\begin{document} 

% set frameboxes to be borderless
\setlength{\fboxsep}{0pt}
\setlength{\fboxrule}{0pt}

\maketitle

\section{Metrics} 
  A metric is an equation for the distance between two points.

  There are two methods for determining the curvature: (1) geometrically and
  (2) by the metric.

  The geometric determination can be done in two ways: (1) measuring the 
  circumference of a circle of expected radius and (2) parallel transport of 
  a vector across the surface around a closed path.

  For \( \phi := k A \implies k = \phi / A \), if \( \phi = \frac{\pi}{2} \) 
  steradians, \( \frac{1}{8} \) of a circle of radius 
  \( R \), 
  \[ k = \frac{\frac{\pi}{2}}{\frac{1}{8} 4 \pi R^2 } = \frac{1}{R^2} \]

  \begin{figure}[!hbt]
    \begin{center}
    \fbox{ \begin{tikzpicture}
      % 2-sphere surface
      \draw (0,0) circle (2cm);
      % perspective-facing circumference
      \draw (-1.404,1.404) arc (180:360:1.404 and 0.2);
      % perspective-hidden circumference
      \draw[dashed] (-1.404,1.404) arc (0:180:-1.404 and 0.2);
      % measured radius
      \draw (0,2) arc (300:360:-.9) node[midway]{\(a\)};
      \draw (0,2) arc (0:180:9) node[midway]{\(a\)};
      % flattened radius
      \draw[dashed] (0,1.404) -- node{\tiny \(R sin
        \left(\frac{a}{R}\right)\)}(1.404,1.404);
      % sphere radius points
      \fill[fill=black] (0,0) circle (1pt);
      \fill[fill=black] (0,2) circle (1pt);
      \fill[fill=black] (1.414,1.414) circle (1pt);
      % projection of triangle
      \draw[dashed] (0,0) -- (0,2);
      \draw[dashed] (.702,0) (0:60);
      \draw[dashed] (0,0) -- node[below]{\(R\)} (1.414,1.414);
      \end{tikzpicture} }
    \end{center}

      \caption{2-sphere of radius \(R\) on which a circle of radius \(a\)
        will be measured to have radius \(a\), which has a flattened radius
        of \(R sin \left( \frac{a}{R} \right) \neq a\)}
      \label{fig:sphere}
  \end{figure}

  Determination by the metric is by Theorema Egregium. Curvature can be
  evaluated for any space, which in 2D is:

  \begin{equation}
    k = \frac{1}{2 g_{11} g_{22}} \{ 
    - \frac{\partial g_{11}}{\partial (x^2)^2} 
    - \frac{\partial g_{22}}{\partial (x^1)^2}
    + \frac{1}{2 g_{11}} [ \frac{\partial g_{11}}{\partial x^1}
      \frac{\partial g_{22}}{\partial x^1} 
    + ( \frac{\partial g_{11}}{\partial x^2} )^2 ]
    + \frac{1}{2 g_{22}} [ \frac{\partial g_{11}}{\partial x^2}
      \frac{\partial g_{22}}{\partial x^2} 
    + ( \frac{\partial g_{22}}{\partial x^1} )^2 ] \} 
  \end{equation}

\section{Laws}

\section{CMB}

This paper is available publicly.\cite{Hayden_Cosmology_Source_Repo}

\pagebreak
\printbibliography

\end{document}

