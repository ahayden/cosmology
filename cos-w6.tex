\documentclass{paper}
\usepackage{times}
\usepackage{geometry}
\geometry{letterpaper, portrait, margin=1in}
\usepackage[utf8]{inputenc}
\usepackage{enumitem,amssymb}
\usepackage{ragged2e}
\usepackage{physics}
\usepackage{siunitx}
\usepackage{float}

\usepackage{caption}
\usepackage[hidelinks]{hyperref}
\usepackage{url}

\usepackage{graphicx}
\usepackage{epstopdf}
\graphicspath{{data/cos-w6}}

\usepackage{biblatex}
\bibliography{refs}

\title{cos-w6 tasks}
\author{ariahayd}
\date{8 March 2022}

\begin{document} 

\maketitle

\begin{enumerate}
    \item % 1.
      \begin{enumerate}
        \item
          The first law of thermodynamics is Eq. \ref{eqn:first-thermo}
          
          \begin{equation}
            \dd{E} + P\dd{V} = \dd{Q}
            \label{eqn:first-thermo}
          \end{equation}
    
          In a universe obeying the cosmological principle, heat is not 
          exchanged so \(\dd{Q} = 0\). Where \(E = \rho c^2 V\), the sum of 
          energy densities in the universe, show 
          \(\dd{\rho} = -(\rho + \frac{P}{c^2}) \frac{\dd{V}}{V}\).
  
          Differentiate \(E = \rho c^2 V\):
          \[ 
            \dd{E} = \dd{\rho}c^2V + \rho c^2\dd{V} 
          \]

          Substitute \(\dd{E}\) into \ref{eqn:first-thermo}:
          \[
            \dd{\rho}c^2V + \rho c^2\dd{V} = -P\dd{V}
          \]
          \[
            \dd{\rho}c^2V = -\left(\rho c^2 + P\right)\dd{V}
            \implies \dd{\rho} 
            = -\left(\rho + \frac{P}{c^2}\right)\frac{\dd{V}}{V}
          \]

        \item
          Derive the fluid equation, Eq. \ref{eqn:fluid} by substituting 
          \(V \propto R^3\):

          \[
            \dd{\rho} = 
            -\left(\rho+\frac{P}{c^2}\right)\frac{\dd{V}}{(\propto R^3)}
          \]

          Differentiate in \(t\):
          \[
            \frac{\dd{\rho}}{\dd{t}} 
            = \frac{\dd{}}{\dd{t}}
            \left[ - \left(\rho+\frac{P}{c^2}\right) 
            \frac{3R^2\dd{R}}{R^3} \right]
            = - \left(\rho + \frac{P}{c^2}\right)
            \frac{3}{R}\frac{\dd{R}}{\dd{t}}
          \]

          \begin{equation}
            \implies
            \dot{\rho} + \left(\rho + \frac{R}{c^2}\right)
            \frac{3}{R} \dot{R} = 0
            \label{eqn:fluid}
          \end{equation}

        \item
          \begin{enumerate}
            \item
              Show that \(\rho \propto R^{-3}\) for a matter-dominated
              (non-relativistic) universe, by using Eq. \ref{eqn:fluid} 
              with \(P=0\).
              \[
                \dot{\rho} + \rho \frac{3}{R} \dot{R} = 0
                \implies \dot{\rho} = - \rho \frac{3}{R} \dot{R}
                \implies \frac{\dot{\rho}}{\rho} = (-3)\frac{\dot{R}}{R}
              \]

              Integrate in \(t\):
              \[
                \frac{\dd{\rho}}{\rho} = (-3)\frac{\dd{R}}{R}
              \]

              Integrate both sides:
              \[
                \ln{\rho} = -3 \ln{R}+\mathcal{C} \implies \rho \propto R^{-3}
              \]

            \item
              Show that \(\rho \propto R^{-4}\) for a radiation-dominated
              universe, by using Eq. \ref{eqn:fluid} with 
              \(P=\frac{1}{3}\rho c^2\).
              \[
                \dot{\rho} + ( \rho +\frac{\frac{1}{3}\rho c^2}{c^2} ) 
                \frac{3}{R} \dot{R} = 0
                \implies
                \dot{\rho} = -\rho(1 + \frac{1}{3})\frac{3}{R}\dot{R}
                \implies
                \frac{\dot{\rho}}{\rho} = (-4)\frac{\dot{R}}{R}
              \]

              Integrate in \(t\):
              \[
                \frac{\dd{\rho}}{\rho} = (-4)\frac{\dd{R}}{R}
              \]

              Integrate both sides:
              \[
                \ln{\rho} = -4 \ln{R}+\mathcal{C} \implies \rho \propto R^{-4}
              \]

          \end{enumerate}

        \item
          Show that \(P = - p c^2\) for a \(\Lambda\) steady-state universe,
          which implies \(\dot{\rho} = 0\). Rewrite Eq. \ref{eqn:fluid}:
          \[
            \dot{\rho} = 0 
            = - \left(\rho + \frac{P}{c^2}\right)\frac{3}{R}\dot{R}
          \]

          \[
            \rho + \frac{P}{c^2} = 0  \implies P = - \rho c^2
          \]

          This result indicates a negative pressure in space due to the 
          vacuum energy density.

      \end{enumerate}

    \item

      \begin{enumerate}

        \item
          Rotation curves of spiral galaxies follow a solid-body 
          gravitational model rather than a decline in rotation away from
          the center of the galaxy, which indicates significant mass away
          from the luminous surface of the galaxy. In this case, the dark
          matter is believed to be non-baryonic because it does not interact 
          with the baryons of the galaxy and so is not made visible.

        \item
          Gravitational lensing of distant quasars across a galactic mass
          is greater than would be expected for the calculated measures of 
          galaxies as the lensing body due to their visible mass alone.

        \item
          The modeled age of the universe requires a density parameter for
          the mass component of energy density to be significantly greater
          than the observed baryonic matter density in order for the 
          expansion of the universe to have delayed self-gravitation
          sufficiently to have reached the age older than the oldest
          observed star clusters.

      \end{enumerate}

    \item
      The density parameter, in Eq. \ref{eqn:omega-i}, of the universe is 
      made up of constituents, \(i\), of different types of energy.
      \begin{equation}
        \Omega_i := \frac{\epsilon_i}{\epsilon_c}
        \label{eqn:omega-i}
      \end{equation}

      Where \(\epsilon_c\) is the critical density for the observed flat
      geometry of space, defined for the current era in Eq. 
      \ref{eqn:epsilon-c}.
      \begin{equation}
        \epsilon_c = \frac{3 c^2 H_o^2}{8 \pi G} = 4.9 \si{GeV m^{-3}}
        \label{eqn:epsilon-c}
      \end{equation}

      It is possible to calculate the number density of representative
      particles from the energy density.

      The current contribution from baryon energy density, 
      \(\Omega_{b,0} \approx 0.04\), can be assumed to be non-relativistic,
      which gives the rest energy of a representative particle within a 
      representative volume as:
      \[
        E = N M_p c^2 = \epsilon_{b,0} V \implies 
        \frac{N}{V} = \frac{\epsilon_{b,0}}{M_p c^2}
      \]
      Where \( \epsilon_{b,0} = \Omega_{b,0}\epsilon_c \) and
      \( M_p c^2 = 938 \si{MeV} \), so
      \[
        \frac{N}{V} = \frac{\Omega_{b,0}\epsilon_c}{M_p c^2}
        = \frac{(0.04)4.9 \si{GeV m^{-3}}}{938 \si{MeV}} = 2.1 \si{m^{-3}}
      \]

      Similarly, the current contribution from radiation energy,
      \(\Omega_{r,0} \approx 5e^{-5}\), can be assumed to be represented
      by photons of the peak frequency given by Wien's law for the
      temperature of the CMB, 
      \(\nu_{peak} = T \times 5.88e^{10} \si{Hz K^{-1}} \implies
      \nu_{peak,T=2.725\si{K}} = 1.60e^{11} \si{Hz} \), giving energy
      within a representative volume as:
      \[
        E = N h \nu_{peak} = \epsilon_{r,0} V \implies
        \frac{N}{V} = \frac{\epsilon_{r,0}}{h \nu_{peak}}
      \]
      Where \( \epsilon_{r,0} = \Omega_{r,0}\epsilon_c \) and
      \( h \nu_{peak} = .0066 \si{eV} \), so 
      \[
        \frac{N}{V} = \frac{\Omega_{r,0}\epsilon_c}{h \nu_{peak}}
        = \frac{(5e^{-5})4.9 \si{GeV m^{-3}}}{.0066 \si{eV}} 
        = 3.7e^{7} \si{m^{-3}}
      \]

    \item
      Rotational velocity implies a central mass of:
      \[
        M = \frac{v^2R}{G}
      \]
      
      For the first case with a \(5\si{kpc}\) aperature implies a central
      mass of:
      \[
        M_{5\si{kpc}} = \frac{(150\si{km s^{-1}}) 5 \si{kpc}}{G}
        = 4.82e^{40} \si{kg} = 2.4e^{10} \si{M_{\odot}}
      \]

      With a measured luminosity within the aperature of 
      \(6.5e^{10}L_{\odot}\), the mass-to-light ratio of the stellar
      population is:
      \[
        \frac{M_{\odot}}{L_{\odot}} 
        = \frac{2.4e^{10}\si{M_{\odot}}}{6.5e^{10}\si{L_{\odot}}} = 0.37
      \]

      At \(50\si{kpc}\), the rotational velocity implies a central mass of:
      \[
        M_{50\si{kpc}} = \frac{(200\si{km s^{-1}}) 50 \si{kpc}}{G}
        = 9.3e^{41} \si{kg} = 4.6e^{11} \si{M_{\odot}}
      \]

      The stellar mass within a \(50 \si{kpc}\) aperature, assuming the same
      mass-to-light ratio calculated for a \(5 \si{kpc}\) aperature, is given
      by:
      \[
        \frac{M_{5 \si{kpc}}}{L_{5 \si{kpc}}} = 
        \frac{M_{stellar}}{L_{50 \si{kpc}}} \implies
        M_{stellar} = 0.37 \times L_{50 \si{kpc}} = 4.8e^{10}M_{\odot}
      \]

      The ratio of central mass measured via rotational velocity to the
      stellar mass estimate is:
      \[
        \frac{M_{50 \si{kpc}}}{M_{stellar}} = 
        \frac{4.6e^{11} M_{\odot}}{4.8e^{10} L_{\odot}} = 9.6
      \]

      The total mass measured at the \(50 \si{kpc}\) aperature by rotational
      velocity implies \(9.6\) times more mass supporting the rotating body
      than in the stellar population within it.

    \item
      \begin{enumerate}
        \item
          The analysis in this paper assumes that 10\% of the mass of the
          galaxy cluster systems are in the plasma component ejected during
          the ram pressure stripping process. 

          Stellar masses in the all galaxies were calculated by summing the
          I-band luminosity for any sources less bright than the baseline
          brightest cluster galaxy and assuming a mass-to-light ratio of
          \(M/L_I = 2 +/- 50\%\)\cite{Clowe_2006}.

          The model assumes that there are no columns of masses along the
          line of sight that could result in a mass sheet degenracy
          The aperature of measurement around each component galaxy is 
          \(100 kpc\), chosen because smaller aperatures have greater
          measurement error in the calculated increase in the size of a
          background galaxy due to anisotropies in the path light travels
          through the potential.  The analysis used a model to create the 
          mass sheet degeneracy of the two-dimensional surface creating the 
          distortion in the presentation of the background objects. 

          The title of the paper is not entirely accurate because there is
          the potential, which is not discussed, for alternative gravity
          formulations that effect masses differently depending on the
          type of matter. In other words, an alternative gravitation could
          effect baryon masses differently from the non-baryon masses. This
          paper does prove that the gravitation effect of dark matter is not
          could to baryonic masses.

          

      \end{enumerate}
      (a) What assumptions are being made in the case of: (i) dark matter being the dominant mass component; and (ii) baryonic matter being the dominant mass component?
(b) Describe the different types of mass measurements being made, and how this favours the dark matter hypothesis.
(c) In your opinion, is the title of the article justified?


\end{enumerate}

This paper is available publicly.\cite{Hayden_Cosmology_Source_Repo}

\pagebreak
\printbibliography

\end{document}

