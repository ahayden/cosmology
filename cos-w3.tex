\documentclass{paper}
\usepackage{times}
\usepackage{geometry}
\geometry{letterpaper, portrait, margin=1in}
\usepackage[utf8]{inputenc}
\usepackage{enumitem,amssymb}
\usepackage{ragged2e}
\usepackage{physics}

\usepackage{caption}
\usepackage[hidelinks]{hyperref}
\usepackage{url}

\usepackage{graphicx}
\usepackage{epstopdf}
\graphicspath{{data/cos-w3}}

\usepackage{biblatex}
\bibliography{refs}

\title{cos-w3 questions}
\author{ariahayd}
\date{15 February 2022}

\begin{document} 

\maketitle

\begin{enumerate}
    \item % 1. 
      For observer \(O\), an event has coordinates: \(\vb{E} = \ev{x,y,z,t}\)

      For observer \(O'\): \(\vb{E'} = \ev{x',y',z',t'}\)

      Show that \(s^2 = (c t)^2 - x^2\) is Lorentz invariant.
      \begin{equation}
        \begin{bmatrix}
          x' \\ y' \\ z' \\ t'
        \end{bmatrix}
        = 
        \begin{matrix}
          \gamma (x - v t) \\ y \\ z \\ \gamma (t - \frac{vx}{c^2})
        \end{matrix}
      \end{equation}

      \[
        s^2 = (ct)^2 - x^2 - y^2 - z^2 
        \stackrel{?}{=}
        s'^2 = (ct')^2 - x'^2 - y'^2 - z'^2 
      \]

      Subtract coordinates in \(y\) and \(z\) as equivalent
      \[ s^2 = (ct)^2 - x^2 \stackrel{?}{=} (ct')^2 - x'^2 \]

      Substitute Lorentz
      \[ (ct')^2 - x'^2 = \left[\gamma\left(t - x \frac{v}{c^2}\right)\right]^2 
      - \left[\gamma\left(x - tv\right)\right]^2 
      = \left(c \gamma t - \frac{xv\gamma}{c}\right)^2 
      - \left(x\gamma - vt\gamma\right)^2 \]

      \[ = 
      \gamma^2\left(\frac{c^4t^2 - 2c^2tvx + x^2v^2}{c^2}\right)^2 
      - \gamma^2\left(\frac{c^2x^2 - 2c^2xtv + c^2t^2v^2}{c^2}\right)^2 \]

      Rewrite \(\gamma^2\)
      \[ \gamma = \frac{1}{\sqrt{1 - \beta^2}}, \beta = \frac{v}{c} \implies
      \gamma^2 = \frac{1}{1-\frac{v^2}{c^2}} = \frac{c^2}{c^2 - v^2}\]

      Substitute \(\gamma^2\)
      \[ = \left(\frac{c^2}{c^2-v^2}\right)
      \left(\frac{c^4t^2 - 2c^2tvx + x^2v^2}{c^2}\right)
      - \left(\frac{c^2}{c^2-v^2}\right)
      \left(\frac{c^2x^2 - 2c^2tvx + c^2t^2v^2}{c^2}\right)
      \]

      \[ = \left(\frac{c^2}{c^2-v^2}\right)
      \left(\frac{c^4t^2 + x^2v^2 - c^2x^2 - c^2t^2v^2}{c^2}\right)
      = \left(\frac{c^4t^2 + x^2v^2 - c^2x^2 - c^2t^2v^2}{c^2 - v^2}\right)
      \]

      \[ = \left(\frac{c^2t^2(c^2 - v^2) - x^2(c^2 - v^2)}{c^2 - v^2}\right)
      = (ct)^2 - x^2 
      \]

    \item % 2. 
      Time is encoded in special relativity as
      \(\dd{s}^2 = c^2\dd{t}^2 - \dd{x}^2 - \dd{y}^2 - \dd{z}^2\), a single 
      factor of \(c^2\).  This is the time component of the worldline of a 
      object moving through space.

      In the Robertson-Walker metric, time is encoded as
      \(\dd{s}^2 = c^2\dd{t}^2 - R^2(t)\dd{S}^2\), a factor of \(c^2\), as 
      well as in the spatial component of the scale of the geometry of the 
      universe, \(R(t)\). This allows space to expand and contract with time
      dependence.

    \item % 3.
      Cosmological distances can be measured by emitted photons, which follow 
      paths through isotropic space to our telescopes.

      \begin{equation}
        \frac{\textrm{Distance apparent in luminosity attenuation}}
        {\textrm{Distance apparent as angular size}}
        = \frac{d_L}{d_A}
      \end{equation}
      
      For \(d_A\), \(d \approx \frac{\textrm{Object diameter,} D}
      {\textrm{Angular size,} \theta}\), the angular size is invariant in an 
      isotropic universe and object diameter is invariant for a 
      gravitationally bound object. \(d_A\) is shifted only as a factor of the 
      distance itself:
      \[ d_{A, observed} = \frac{d_{A0}}{1 + z} \]

      For \(d_L\), \(d^2 = \frac{L}{4 \pi F}\) where flux is the photon power
      per solid angle attenuated by a factor in distance itself and also in 
      the slowed arrival rate of photon wave packets, so
      \[ d_L^2 = \frac{L}{4 \pi F} = 
      \frac{L}{4 \pi (\frac{L}{4 \pi d_{L0}^2 r^2} 
      \frac{R_e}{d_{L0}} \frac{R_e}{d_{L0}} ) } \]

      \[ d_L = d_{L0}(1 + z) \]

      So, \(\frac{d_L}{d_A}\) is the ratio of these two factors of \(1 + z\):
      \[ \frac{d_L}{d_A} = \frac{d_{L0}(1 + z)}{\frac{d_{A0}}{(1+z)}} 
      \propto (1 + z)^{2}\]

   \item % 4.
      Proper distance is the separation between two objects moving with the
      Hubble Flow. The rate of change in the proper distance between two
      comoving observers is the geodesic between the points of observation
      \(ds^2 = c^2dt^2 - R^2(t)dS^2\), where \(c^2t^2 = 0 \) for the case of
      fundamental observers. Then, the geodesic is simplified as:

      \begin{equation}
        \dd{s}^2 = - {R}^2(t)\dd{S}^2 \implies \dd{s} \propto \abs{R}
      \end{equation}

      Where \(R\) is the time dependent scale factor of the intrinsic 
      geometry of the universe, \(dS\) which is independent of time. 
      The Hubble Law is the rate of expansion of \(ds\) assuming isotropy
      and homogeneity, implying constant variance in \(R\) with \(t\).

      \[ \frac{\dd{s}}{\dd{t}} = \dot{R} \]

      The Hubble Law is not accurate on scales smaller than 10 Mpc because 
      gravitational effects cause peculiar motions that deviate from the 
      Hubble Flow. On large scales, greater than 1 Gpc, the era of study
      becomes dominated by matter and radiation rather than the expansion
      forces of the Hubble Law.

\end{enumerate}

This paper is available publicly.\cite{Hayden_Cosmology_Source_Repo}

\pagebreak
\printbibliography

\end{document}

